\documentclass{ctexart}
\usepackage{graphicx}
\usepackage{verbatim}
\usepackage{textcomp}
\usepackage{tikz}
 
\begin{document}
\usetikzlibrary{shapes,arrows}

\tikzstyle{block} = [draw, fill=blue!20, rectangle, 
    minimum height=3em, minimum width=3em]
\tikzstyle{sum} = [draw, fill=blue!20, circle, node distance=1cm]
\tikzstyle{input} = [coordinate]
\tikzstyle{output} = [coordinate]
\tikzstyle{pinstyle} = [pin edge={to-,thin,black}]

% The block diagram code is probably more verbose than necessary
\begin{figure}
\centering

\begin{tikzpicture}[auto, node distance=2cm,>=latex']
    \node [input, name=input] {};
    \node [sum, right of=input] (sum) {};
    \node [block, right of=sum] (controller) {\(k_p + \frac{k_I}{s}\)};
    \node [sum, right of=controller, pin={[pinstyle]above:\(y_u\)}, node distance=2cm] (sum2) {};
    \node [block, right of=sum2, node distance=2cm] (integral) {\(\int\)};
    % We draw an edge between the controller and system block to 
    % calculate the coordinate u. We need it to place the measurement block. 
    \draw [->] (controller) -- node[name=u] {} node[pos=0.8, below] {+} (sum2);
    \draw [->] (sum2) -- node[name=u] {} (integral) node[pos=0.1, above] {+};
    \node [output, right of=integral] (output) {};

    % Once the nodes are placed, connecting them is easy. 
    \draw [draw,->] (input) -- node {$y_x$} (sum) node[pos=0.9, below]{+};
    \draw [->] (sum) -- node {} (controller);
    \draw [->] (integral) -- node [name=out] {$\hat x$}(output);
    \draw [->] (out) |- (1,-1.5) -| node[pos=0.99] {} node [near end] {} (sum) node[pos=0.92, right]{-};

\end{tikzpicture}
\caption{}
\label{fig:complement_filter_control_diagram}
\end{figure}

\end{document}