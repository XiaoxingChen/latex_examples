%%%%%%%%%%%%%%%%%%%%%%%%%%%%%%%%%%%%%%%%%%%%%%%%%%%%%%%%%%%%%%%
%
% Welcome to Overleaf --- just edit your LaTeX on the left,
% and we'll compile it for you on the right. If you give
% someone the link to this page, they can edit at the same
% time. See the help menu above for more info. Enjoy!
%
% Note: you can export the pdf to see the result at full
% resolution.
%
%%%%%%%%%%%%%%%%%%%%%%%%%%%%%%%%%%%%%%%%%%%%%%%%%%%%%%%%%%%%%%%
% Title: Block diagram of Third order noise shaper in Compact Disc Players
% Author: Ramón Jaramillo

\documentclass{ctexart}
\usepackage{graphicx}
\usepackage{verbatim}
\usepackage{textcomp}
\usepackage{tikz}
\usepackage{amsfonts} 
 
\begin{document}
 
\usetikzlibrary{shapes,arrows}
\tikzstyle{block} = [draw, fill=blue!20, rectangle, 
    minimum height=3em, minimum width=3em]
\tikzstyle{sum} = [draw, fill=blue!20, circle, node distance=1cm]
\tikzstyle{input} = [coordinate]
\tikzstyle{output} = [coordinate]

% The block diagram code is probably more verbose than necessary

\begin{tikzpicture}[auto, node distance=2cm,>=latex',scale=0.5, transform shape]
    \node [input, name=input] {R};
    \node [block, right of=input] (diff) {\(\hat R ^ TR\)};
    \node [block, right of=diff] (tangent_space_error) {\(\mathbb{P}_a (\tilde{R})\)};
    \node [block, right of=tangent_space_error] (gain) {K};
    \node [sum, right of=gain, node distance=2cm] (sum) {};
    \node [block, right of=sum] (kinematics) {\(\dot{\hat{R}} = A \hat{R}\)};
    \node [output, right of=kinematics, node distance=3cm] (output) {\(\hat{R}\)};
    \node [block, below of=diff](inver_of_error) {\(\hat{R}^T\)};
    \node [block, above of=sum] (tangent_space_velocity) {\((R\Omega)_\times\)};
    \node [block, above of=tangent_space_velocity] (velocity) {\(R\Omega\)};

    \node [input, name=input_angle, left of= velocity, node distance=8cm] {};
    \node [input, name=input_velocity, above of= velocity] {};

    % % Once the nodes are placed, connecting them is easy. 
    \draw [draw,->] (input) -- node {R} (diff);
    \draw [->] (diff) -- node {} (tangent_space_error) node[pos=0.5, above] {\(\tilde{R}\)};
    \draw [->] (tangent_space_error) -- node {} (gain) ;
    \draw [->] (gain) -- node {} (sum) node[pos=0.8, below] {+};
    \draw [->] (sum) -- node {} (kinematics) ;
    \draw [->] (kinematics) -- node [name=mid_out] {\(\hat R\)} (output) ;
    \draw [->] (mid_out) |- (7,-2) -- (inver_of_error){};
    \draw [->] (inver_of_error) -- (diff){};

    \draw [draw,->] (input_angle) -- node {} (velocity) node[pos=0.1, above, name=error] {R};
    \draw [draw,->] (input_velocity) -- node {\(\Omega\)} (velocity);
    \draw [->] (tangent_space_velocity) -- (sum){} node[pos=0.8, left] {+};
    \draw [->] (velocity) -- (tangent_space_velocity){} ;

\end{tikzpicture}
 
\end{document}



